\chapter{Results and Discussion}
\label{sec:results}

\section{Results}
\begin{table}[H]
    \centering
    \caption{Best performing model in terms of Cohen's Kappa run on a testing data set in a specific combination of features per personality trait for each modality}
    \begin{tabularx}{\textwidth}{l|X|l|l}
    \hline
        \textbf{Trait} & \textbf{Combination} & \textbf{Model} & \textbf{Test Kappa} \\ \hline
        \multicolumn{4}{l}{\textbf{Unimodal}} \\ \hline
        \multicolumn{4}{l}{\textbf{Most Agreement}} \\ \hline
        Openness & Resnet50 + HSV & Logistic Regression (GPU) & 0.2762 \\ \hline
        Conscientiousness & Resnet50 + Imagga & XGBoost & 0.1174 \\ \hline
        Extraversion & Resnet50 + HSV & XGBoost & 0.2632 \\ \hline
        Agreeableness & Resnet50 + HSV & SVM (GPU) & 0.0444 \\ \hline
        Neuroticism & Resnet50 + HSV & SVM (GPU) & 0.0037 \\ \hline
        \multicolumn{4}{l}{\textbf{Most Disagreement}}\\ \hline
        Openness & Imagga + HSV & XGBoost & -0.0133 \\ \hline
        Conscientiousness & N/A & N/A & N/A \\ \hline
        Extraversion & N/A & N/A & N/A \\ \hline
        Agreeableness & Resnet50 + Imagga + HSV & Logistic Regression (GPU) & -0.1214 \\ \hline
        Neuroticism & Resnet50 + Imagga + HSV & SVM (GPU) & -0.0788 \\ \hline
    \end{tabularx}
    \label{bestcohenmodel}
\end{table}
\begin{table}[H]
    \centering
    \caption{Best performing model in terms of Cohen's Kappa run on a testing data set in a specific combination of features per personality trait for each modality}
    \begin{tabularx}{\textwidth}{l|X|l|l}
    \hline
        \textbf{Trait} & \textbf{Combination} & \textbf{Model} & \textbf{Test Kappa} \\ \hline
        \multicolumn{4}{l}{\textbf{Bimodal}} \\ \hline
        \multicolumn{4}{l}{\textbf{Most Agreement}} \\ \hline
        Openness & Resnet50 + HSV, Metadata & Logistic Regression (GPU) & 0.2602 \\ \hline
        Conscientiousness & TFIDF + Word2Vec, Metadata & SVM (GPU) & 0.2451 \\ \hline
        Extraversion & Word2Vec, HSV & Logistic Regression (GPU) & 0.3684 \\ \hline
        Agreeableness & Word2Vec, Metadata & SVM (GPU) & 0.2129 \\ \hline
        Neuroticism & Word2Vec, HSV & XGBoost & 0.0749 \\ \hline
        \multicolumn{4}{l}{\textbf{Most Disagreement}} \\ \hline
        Openness & TFIDF, Imagga & XGBoost & -0.0155 \\ \hline
        Conscientiousness & Resnet50, Metadata & XGBoost & -0.037 \\ \hline
        Extraversion & N/A & N/A & N/A \\ \hline
        Agreeableness & Resnet50 + Imagga, Metadata & Logistic Regression (GPU) & -0.1032 \\ \hline
        Neuroticism & Imagga + HSV, Metadata & XGBoost & -0.1259 \\ \hline
    \end{tabularx}
    \label{bestcohenmodel}
\end{table}
\begin{table}[H]
    \centering
    \caption{Best performing model in terms of Cohen's Kappa run on a testing data set in a specific combination of features per personality trait for each modality}
    \begin{tabularx}{\textwidth}{l|X|l|l}
    \hline
        \textbf{Trait} & \textbf{Combination} & \textbf{Model} & \textbf{Test Kappa} \\ \hline
        \multicolumn{4}{l}{\textbf{Multimodal}}\\ \hline
        \multicolumn{4}{l}{\textbf{Most Agreement}} \\ \hline
        Openness & Word2Vec, Resnet50 + Imagga + HSV, Metadata & XGBoost & 0.2267 \\ \hline
        Conscientiousness & TFIDF + Word2Vec, HSV, Metadata & SVM (GPU) & 0.2442 \\ \hline
        Extraversion & TFIDF + Word2Vec, Resnet50, Metadata & XGBoost & 0.2807 \\ \hline
        Agreeableness & Word2Vec, HSV, Metadata & SVM (GPU) & 0.1979 \\ \hline
        Neuroticism & Word2Vec, HSV, Metadata & XGBoost & 0.065 \\ \hline
        \multicolumn{4}{l}{\textbf{Most Disagreement}} \\ \hline
        Openness & N/A & N/A & N/A \\ \hline
        Conscientiousness & N/A & N/A & N/A \\ \hline
        Extraversion & N/A & N/A & N/A \\ \hline
        Agreeableness & Word2Vec, Resnet50, Metadata & XGBoost & -0.0201 \\ \hline
        Neuroticism & TFIDF, Imagga, Metadata & SVM & -0.1218 \\ \hline
    \end{tabularx}
    \label{bestcohenmodel}
\end{table}
\section{Discussion}

\subsection{Unimodal}
In the case of TF-IDF, using all tags rather than only the ones that fall in the 1,500 most common generally performed better, with the exception of Agreeability. This shows that using all available information and taking less common nuances into account likely improves personality prediction. Between Word2Vec’s two averaging strategies, the one where post composition was prioritized (“Posts Weighed Equally”) performed better than the alternative, with the exception of Neuroticism; however, the difference  for that trait is minor. This could mean that taking overall post composition and sentiment into account, rather than just individual word usage, may reveal aspects of personality that are useful in its prediction. The combination of these two text features for personality prediction did not show any improvement over the individual results, with some results, most notably TF-IDF’s Neuroticism and Word2Vec’s Extraversion, performing better on their own.

Imagga and Resnet50 both had no notable results besides those for Consciousness and Extraversion, exemplifying a pattern among most, if not all, models in this study that these are the most accurately predicted personality traits. HSV results underperformed across the board, with the best results coming out just barely over 0, if at all. However, when combined with Resnet50, they yield some of the most notable results of the individual image modality, implying that this feature may still contain valuable personality data. However, this may also be because Resnet50 already had a strong baseline performance on its own, as it was not a major improvement from those results. 


\subsection{Bimodal}
From bimodal results, we can once more observe that Extraversion remains the easiest trait to predict, followed by Openness and Conscientiousness. Extraversion demonstrates the most consistent scores across combinations, suggesting that it exhibits stronger behavioral signals in social media compared to other traits. While Openness and Conscientiousness also showed promising results, they were less consistent. Agreeableness, while showing a few good results, was moderate overall. Lastly, Neuroticism once more proved to be the most difficult, with results clustering around zero or negative values. This pattern is consistent with prior personality computing studies, where internally experienced traits such as Neuroticism are less observable from external user-generated data.

In general, combining image and text modalities yields limited performance gains compared to unimodal performance. However, there was one significant improvement: the Word2Vec + HSV combination for Conscientiousness outperformed its parts individually, achieving a 0.2233 compared to their respective individual 0.1102 and 0.0115 scores. This suggests some complementary signal between semantic text features and low-level color statistics. However, this value did not outperform the overall best unimodal Conscientiousness model (0.2271 from TF-IDF) nor the overall best bimodal combinations discussed later. 

From the combination of the text and metadata modalities, some of the best-performing models among unimodal and bimodal models in this study are obtained. The combination of TF-IDF, Word2Vec, and metadata produced the best performance for Conscientiousness so far, with a kappa of 0.2451. The combination of Word2Vec and metadata also gave the current best model for Agreeability, with a kappa of 0.2129. These findings indicate that behavioral metadata provides informative personality cues that complement linguistic features. The improvement also suggests that user activity patterns may encode personality-relevant signals not fully captured by content alone.

The image-metadata combinations showed some improvement over their individual parts. Both ResNet50 + metadata and ResNet + HSV + metadata achieve $\kappa$ = 0.2807 for Extraversion. However, these still fall short of the best unimodal Extraversion model (Word2Vec, $\kappa$ = 0.3684), reinforcing the stronger predictive power of textual representations. We can also see from these combinations, as well as from unimodal performance, that the Conscientiousness trait appears to be more predictable with text, with most scores close to zero. The best score, 0.11, is just average compared to models that use text.

Overall, the results suggest that text—particularly semantic embeddings such as Word2Vec—remains the strongest modality for personality prediction in this dataset. While multimodal fusion shows potential, especially in Word2Vec + HSV and metadata-enhanced models, the gains are inconsistent. This may indicate partial redundancy between modalities or limitations in the current early-fusion strategy. Future work may benefit from more advanced fusion architectures or fine-tuned visual representations to better exploit cross-modal complementarities.

\subsection{Multimodal}

\section{Model Performance Analysis}
The performance of the classification models—Logistic Regression (LR), Support Vector Machine (SVM), and XGBoost—was evaluated across various unimodal, bimodal, and multimodal feature combinations for each of the Big Five personality traits. The primary metric for evaluation was Cohen’s Kappa ($\kappa$), which accounts for the possibility of agreement occurring by chance.

\subsection{Performance by Trait}
Model performance varied significantly depending on the trait and the specific combination of features utilized.

\begin{itemize}
	\item \textbf{Extraversion:} This trait generally showed the strongest and most consistent predictive performance. In bimodal visual combinations such as Combination U7 (ResNet + Imagga), XGBoost achieved a test kappa of 0.2105. Performance peaked in multimodal settings, specifically Combination M1 (TF-IDF + ResNet + Metadata) using SVM, which yielded a test kappa of 0.2456.
	\item \textbf{Openness:} Predictive accuracy for Openness was highest when visual features were included. Combination U8 (ResNet + HSV) achieved a test kappa of 0.2762 with Logistic Regression, marking one of the highest scores across all experiments.
	\item \textbf{Conscientiousness:} This trait saw moderate performance. Combination B5 (TF-IDF + HSV) reached a test kappa of 0.1537 using Logistic Regression, while the addition of metadata in Combination M3 (TF-IDF + HSV + Metadata) slightly improved this to 0.1569.
	\item \textbf{Agreeableness and Neuroticism:} These traits proved the most challenging to predict. Results often yielded low or negative kappa values, suggesting that the selected features may not fully capture the nuances of these dimensions within the Filipino Instagram context. For instance, Neuroticism in Combination M1 resulted in a kappa of -0.0907.
\end{itemize}

\subsection{Comparison of Feature Fusion Approaches}
A key objective of this study was to determine if multimodal fusion improves prediction over unimodal or bimodal data.

\begin{itemize}
	\item \textbf{Bimodal Visual Combinations:} Combining different types of visual features (e.g., ResNet for deep features and HSV for color distribution) often improved performance over single-source visual sets. Combination U8 (ResNet + HSV) significantly outperformed U9 (Imagga + HSV) for both Openness and Extraversion.
	\item \textbf{Multimodal Fusion:} The integration of text (TF-IDF), images (ResNet/HSV), and account metadata generally provided more stable results across traits. Combination M4, which integrated TF-IDF, ResNet, Imagga, and Metadata, achieved a test kappa of 0.1264 for Openness. 
	\item \textbf{Overfitting Observations:} Despite high performance in certain categories, some bimodal text-image combinations faced issues with dimensionality (exceeding 38,000 features). This led to instances of overfitting, where high validation kappas (e.g., 0.8372 for Neuroticism in some XGBoost runs) did not translate to the test set, which remained negative.
\end{itemize}

\subsection{Algorithm Comparison}
The three models exhibited distinct performance characteristics across the dataset:

\begin{itemize}
	\item \textbf{XGBoost:} Frequently the top performer for Extraversion and Openness. It demonstrated a superior ability to handle complex, high-dimensional feature spaces, particularly in bimodal visual tasks.
	\item \textbf{Logistic Regression:} Performed best when metadata or HSV features were involved. This suggests that these specific features may have a more linear relationship with certain personality traits like Conscientiousness.
	\item \textbf{SVM:} Highly effective in specific multimodal configurations. It achieved the highest overall score for Extraversion in combinations M1 and M5, reaching a kappa of 0.2456.
\end{itemize}
