\chapter{Results and Discussion}
\label{sec:results}

\section{Results}
\begin{table}[H]
    \centering
    \caption{Best performing models per personality trait with its feature combination}
    \label{tab:best_models}
    \begin{tabularx}{\textwidth}{l|X|X|l}
        \hline
        \textbf{Trait} & \textbf{Feature Type} & \textbf{Model} & \textbf{Test Kappa} \\ \hline
        Openness & Resnet50 + HSV & Logistic Regression (GPU) & 0.2762 \\ \hline
        Conscientiousness & TFIDF + Word2Vec, Metadata & SVM (GPU) & 0.2451 \\ \hline
        Extraversion & Word2Vec, HSV & Logistic Regression (GPU) & 0.3684 \\ \hline
        Agreeableness & Word2Vec, Metadata & SVM (GPU) & 0.2129 \\ \hline
        Neuroticism & TF-IDF (All) & SVM & 0.1615 \\ \hline
    \end{tabularx}
\end{table}

In the Table \ref{tab:best_models}, among the personality traits, Extraversion has Logistic Regression, run in GPU, as the best performing model with the highest kappa score, boasting 0.3684, with the feature combination Word2Vec and HSV. On the other hand, Neuroticism has XGBoost as the worst performing model compared to the models of each trait, with the kappa score of 0.0749, with the feature combination Word2Vec and HSV.

Openness, Extraversion and Neuroticism are consistent with their feature type combination, being Word2Vec and HSV. However, Conscientiousness only has TF-IDF, Word2Vec, and Metadata as the feature type combination used by the best model, which is SVM, under Conscientiousnes with the kappa score of 0.2451. Agreeableness has Word2Vec and Metadata combination, which are the text feature and a metadata feature.

\begin{table}[h]
	\centering
	\caption{Unimodal Text Results (Cohen's $\kappa$)}
	\label{tab:unimodal_text_performance}
	\begin{tabular}{llccccc}
		\hline
		\textbf{Feature} & \textbf{Variant} & \textbf{O} & \textbf{C} & \textbf{E} & \textbf{A} & \textbf{N} \\
		\hline
		\multirow{2}{*}{TF-IDF} % Added {2} and {*}
		& 1.5k & 0.1005 & 0.0969 & 0.1754 & 0.1217 & -0.0242 \\
		& All  & 0.1531 & 0.2271 & 0.2456 & 0.0200 & 0.1615 \\
		\hline
		\multirow{2}{*}{Word2Vec} % Added {2} and {*}
		& Words Weighted Equally & 0.1381 & 0.0297 & 0.1754 & 0.1174 & 0.0000 \\
		& Posts Weighted Equally & 0.2267 & 0.1102 & 0.3684 & 0.1620 & -0.0447 \\
		\hline
		Combined & -- & 0.1706 & 0.1960 & 0.1404 & 0.1206 & 0.0188 \\
		\hline
	\end{tabular}
\end{table}

Table ~\ref{tab:unimodal_text_performance}, outlines the performance of unimodal text results, namely for TF-IDF, Word2Vec, and their combinations for all personality traits. This table also reports the performances of the two variations of the text models: only the top 1,500 most frequent tags of TF-IDF versus all tags, and the “Words Weighted Equally” approach of Word2Vec versus the “Posts Weighted Equally” approach.

In the case of TF-IDF, using all tags rather than only the top 1,500 generally performed better—having kappa differences ranging from 0.0526 to 0.1857—with the exception of Agreeability, whose 1,500 tag variation achieved a 0.1217, a 0.1017 difference from the other. This shows that using all available information and taking less common nuances into account likely improves personality prediction. Between Word2Vec’s two averaging strategies, on the other hand, the one that prioritized post composition (“Posts Weighed Equally”) performed better than the alternative, with kappa score differences ranging from 0.0446 to 0.1930. The exception is for Neuroticism; however, the difference for that trait is minor and the result unremarkable ($\kappa$ = 0.00). This could mean that taking overall post composition and sentiment into account, rather than just individual word usage, may reveal aspects of personality that are useful in its prediction. 

The combination of the two text features did not improve on the individual results, with some results, most notably TF-IDF’s Neuroticism and Word2Vec’s Extraversion, performing better on their own. Most combination scores came between the best and worst scores of their respective parts, with the exception of Extraversion, whose combination performed worst overall. These results show that TF-IDF and Word2Vec likely share overlapping personality information, which does not broaden our understanding but instead creates noise.

\begin{table}[H]
    \centering
    \caption{Best performing model after learning unimodal features in terms of Cohen's Kappa running on a testing data, set in a specific combination of features per personality trait}
    \label{tab:uni_kappa}
    \begin{tabularx}{\textwidth}{l|X|X|l}
    \hline
        \textbf{Trait} & \textbf{Combination} & \textbf{Model} & \textbf{Test Kappa} \\ \hline
        Openness & Resnet50 + HSV & Logistic Regression (GPU) & 0.2762 \\ \hline
        Conscientiousness & Resnet50 + Imagga & XGBoost & 0.1174 \\ \hline
        Extraversion & Resnet50 + HSV & XGBoost & 0.2632 \\ \hline
        Agreeableness & Resnet50 + HSV & SVM (GPU) & 0.0444 \\ \hline
        Neuroticism & Resnet50 + HSV & SVM (GPU) & 0.0037 \\ \hline
    \end{tabularx}
    \label{bestcohenmodel}
\end{table}

In Table ~\ref{tab:uni_kappa}, Openness has Logistic Regression as the best model that used ResNet50 and HSV with the test kappa of 0.2762. Then, Neuroticism has very near zero test kappa of 0.0037 for SVM that uses Resnet50 and HSV.

All of the traits in Table \ref{tab:uni_kappa} except Consciousness have Resnet50 and HSV as the best feature combination, while Resnet50 and Imagga performed the best for Consciousness. Openness, Conscientiousness, and Extraversion have best performing models with test kappa scores that exceed 0.1 threshold, while Agreeableness and Neuroticism have their best models perform weak with their test kappa less than 0.1. Neuroticism has the best model with the least scores among the other personality traits.

Aside from Logistic Regression, two models occur twice among the personality traits. Conscientiousness and Extraversion have XGBoost as the best performing model, while Agreeable and Neuroticism have SVM.

\begin{table}[H]
    \centering
    \caption{Best performing model after learning bimodal features in terms of Cohen's Kappa run on a testing data set in a specific combination of features per personality trai}
    \label{tab:bi_kappa}
    \begin{tabularx}{\textwidth}{l|X|X|l}
    \hline
        \textbf{Trait} & \textbf{Combination} & \textbf{Model} & \textbf{Test Kappa} \\ \hline
        Openness & Resnet50 + HSV, Metadata & Logistic Regression (GPU) & 0.2602 \\ \hline
        Conscientiousness & TFIDF + Word2Vec, Metadata & SVM (GPU) & 0.2451 \\ \hline
        Extraversion & Word2Vec, HSV & Logistic Regression (GPU) & 0.3684 \\ \hline
        Agreeableness & Word2Vec, Metadata & SVM (GPU) & 0.2129 \\ \hline
        Neuroticism & Word2Vec, HSV & XGBoost & 0.0749 \\ \hline
    \end{tabularx}
    \label{bestcohenmodel}
\end{table}

In Table \ref{tab:bi_kappa}, Extraversion has Logistic Regression as the best performing model with the test kappa of 0.3684 among the traits, which uses Word2Vec and HSV. Neuroticism has XGBoost as the worst performing model with the test kappa of 0.0749 among the traits, which uses Word2Vec and HSV. Openness, Conscientiousness and Agreeable have test kappas close to each other with 0.2602, 0.2451 and 0.2129, respectively. Openness was Logistic  Regression as the best performing model.

For feature type combination, Extraversion and Neuroticism have the same combination with Word2Vec and HSV in Table \ref{tab:bi_kappa}, except Extraversion has Logistic Regression and Neuroticism has XGBoost. The rest of the traits have unique feature combination: Openness has Resnet50, HSV and metadata; Conscientiousness has TF-IDF, Word2Vec, and metadata; and Agreeable has Word2Vec and Metadata. 

The models except XGBoost appear at least two times in Table \ref{tab:bi_kappa}, which are Logistic Regression and SVM. Openness and Extraversion have Logistic Regression as the best model, while Conscientiousness and Agreeable have SVM as the best model.

\begin{table}[H]
    \centering
    \caption{Best performing model after learning multimodal features in terms of Cohen's Kappa run on a testing data set in a specific combination of features per personality trait for each modality}
    \label{tab:mm_kappa}
    \begin{tabularx}{\textwidth}{l|X|l|l}
    \hline
        \textbf{Trait} & \textbf{Combination} & \textbf{Model} & \textbf{Test Kappa} \\ \hline
        Openness & Word2Vec, Resnet50 + Imagga + HSV, Metadata & XGBoost & 0.2267 \\ \hline
        Conscientiousness & TFIDF + Word2Vec, HSV, Metadata & SVM (GPU) & 0.2442 \\ \hline
        Extraversion & TFIDF + Word2Vec, Resnet50, Metadata & XGBoost & 0.2807 \\ \hline
        Agreeableness & Word2Vec, HSV, Metadata & SVM (GPU) & 0.1979 \\ \hline
        Neuroticism & Word2Vec, HSV, Metadata & XGBoost & 0.065 \\ \hline
    \end{tabularx}
    \label{bestcohenmodel}
\end{table}

In Table \ref{tab:mm_kappa}, Extraversion has SVM as the best performing model that used TF-IDF, Word2Vec, Resnet50 and metadata with the kappa score of 0.2807, while Neuroticism has XGBoost as the worst performing model that used Word2Vec, HSV and metadata with the kappa score of 0.065.

Across all personality traits, Word2Vec and metadata appear all times in Table \ref{tab:mm_kappa}. HSV is the next most occurring, where only Extraversion does not use HSV. Then TF-IDF comes in to appear in Conscientiousnes and Extraversion. Then ResNet50 is used in Openness and Extraversion. Lastly, Imagga appears once in Openness.

XGBoost is used three times in Table \ref{tab:mm_kappa}, whereas SVM appears twice in the same table. Openness, Extraversion and Neuroticism use XGBoost as the best performing model, while Conscientiousness and Agreeableness use SVM also as the best performing model.

\section{Discussion}

\subsection{Multimodality}
When combining modalities, the performance gains are generally limited and inconsistent relative to unimodal performance; however, certain multimodal configurations do demonstrate meaningful improvements over their individual components. Several bimodal models outperform their respective single-modality counterparts. For instance, the Word2Vec + HSV combination for Conscientiousness achieves a kappa of 0.2233, exceeding the performance of Word2Vec alone ($\kappa$ = 0.1102) and HSV alone ($\kappa$ = 0.0115). This is one of several cases in which bimodal feature fusion leads to improved performance, suggesting that complementary personality signals emerge when modalities are combined. This effect is most clearly illustrated by two models: (1) the TF-IDF + Word2Vec + metadata combination and (2) the Word2Vec + metadata combination. These models achieve the best performance for Conscientiousness ($\kappa$ = 0.2451) and Agreeableness ($\kappa$ = 0.2129), respectively.

Trimodal models exhibited patterns similar to bimodal configurations: performance improvements were observed with the addition of modalities, but the gains were generally modest. One notable example is the Word2Vec + ResNet + Imagga combination, whose kappa increased from 0.1053 to 0.2105 when metadata was incorporated, representing a 0.1052 improvement. However, unlike the bimodal results, none of the trimodal models achieved the best performance for any personality trait in this study. This outcome may reflect limitations in the specific feature combinations explored rather than the ineffectiveness of higher modality fusion, as increasing modality count has already demonstrated potential for performance gains.


\subsection{Trait and feature performance}
An examination of the top models for each trait reveals distinct modality patterns across personality dimensions. For Openness, ResNet-based features clearly dominate, as the highest-performing models consist either of ResNet alone or ResNet combined with metadata and HSV features. Given that ResNet contributes 2,048 features, compared to only 21 from HSV and 2 from metadata, these multimodal configurations are likely driven primarily by the high-dimensional visual representations. This interpretation is further supported by the fact that unimodal ResNet achieves the strongest performance for this trait. The best-performing non-ResNet model is unimodal Word2Vec ($\kappa$ = 0.2267), which also appears frequently among the top Openness models below the ResNet-dominated group. This pattern suggests that while visual representations are the primary signal for Openness in this dataset, semantic textual features provide a secondary yet meaningful contribution.

For Conscientiousness, the highest score comes from the bimodal TF-IDF + Word2Vec + metadata model ($\kappa$ = 0.2451). Most top configurations involve combinations of TF-IDF, Word2Vec, metadata, and HSV, with at least one textual representation always present.  Although metadata and HSV have relatively low dimensionality, their frequent appearance among high-performing configurations instead of the counterpart configurations that lack them suggests that they contribute useful complementary information rather than merely adding noise. Additionally, given the substantially higher dimensionality of textual features, it can be concluded that Conscientiousness is primarily text-driven, with metadata and HSV acting as supporting signals rather than dominant ones. Compared to Openness, image features are largely absent from the top models.

Extraversion is the overall best-performing trait, with unimodal Word2Vec (“Posts Weighted Equally”) achieving the highest kappa score of 0.3684. Similar to the Openness–ResNet pattern, Word2Vec performs best independently and is followed by combinations with HSV and metadata. ResNet also appears in several high-ranking combinations after the strongest Word2Vec-based models, indicating that visual features can provide additional gains. TF-IDF appears occasionally among top models but generally trails behind Word2Vec. 

For Agreeableness, the top-performing model is Word2Vec + metadata ($\kappa$ = 0.2129). Word2Vec appears in nearly all top configurations, typically combined with metadata, TF-IDF, or HSV. This pattern closely resembles Conscientiousness, where text forms the backbone of performance, and lower-dimensional features such as metadata and HSV frequently accompany it. As with Conscientiousness, these smaller feature sets appear to provide complementary value despite their limited dimensionality.

In contrast, Neuroticism shows a markedly different pattern. The best-performing model is unimodal TF-IDF ($\kappa$ = 0.1615), substantially outperforming the next highest model, Word2Vec + HSV ($\kappa$ = 0.0749). Beyond this leading result, most models cluster near zero. This trait has proven difficult to predict, no matter the combination of features or modalities.

From these rankings, some general observations can be made. Metadata and HSV frequently appear in high-performing combinations despite their low dimensionality. This raises two possibilities: either they add minimal noise and simply “ride along” with stronger high-dimensional features, or they provide compact but meaningful complementary signals. The substantial $\kappa$ increase observed in earlier experiments when adding only the two metadata features (+0.1052) supports the latter interpretation in at least some cases. Conversely, Imagga—despite having the largest dimensionality (36,398 features)—rarely appears among top-performing models. This suggests that high dimensionality alone does not guarantee predictive value and may instead introduce noise that limits its contribution. It may also suggest that Imagga does not carry a valuable amount of personality data.


\section{Model Performance Analysis}
The performance of the classification models—Logistic Regression (LR), Support Vector Machine (SVM), and XGBoost — was evaluated to determine the best configurations for each of the Big Five personality traits. The primary metric for evaluation was Cohen’s Kappa ($\kappa$), which measures the agreement between predicted and actual labels while accounting for chance.

\subsection{Overall Best Performing Models}
Table~\ref{tab:best_overall} summarizes the winning configurations for each trait. Unlike earlier broad sweeps, these models represent the optimal balance of modalities and algorithm selection.

\begin{table}[H]
	\centering
	\caption{Best Performing Models and Hyperparameters per Personality Trait}
	\label{tab:best_overall}
	\begin{tabularx}{\textwidth}{@{} l X X c @{}}  % X column allows wrapping in Model column
		\toprule
		\textbf{Trait} & \textbf{Combination} & \textbf{Model (Params)} & \textbf{Kappa} \\
		\midrule
		O          & ResNet50 + HSV                         & LR ($C=0.1$)                                  & 0.2762 \\
		C & TF-IDF + W2V + Metadata                 & SVM ($C=1.0$, $\gamma=0.0001$, RBF)           & 0.2451 \\
		E      & Word2Vec + HSV                           & LR ($C=0.3594$)                               & 0.3684 \\
		A      & Word2Vec + HSV + Metadata                & SVM ($C=10.0$, $\gamma=0.0001$, RBF)          & 0.1979 \\
		N       & TF-IDF                      & SVM ($C=1.0$, $\gamma=0.0001$, RBF)           & 0.1615 \\
		\bottomrule
	\end{tabularx}
\end{table}

\subsection{Trait-Specific Analysis and Interpretable Features}
% Table 1: Top Pearson Correlations by Trait
\begin{table}[htbp]
	\centering
	\caption{Top Pearson Correlations by Trait (positive and negative)}
	\label{tab:pearson}
	\begin{tabular}{@{} l l c l c @{}}
		\toprule
		\textbf{Trait} & \multicolumn{2}{c}{\textbf{Positive}} & \multicolumn{2}{c}{\textbf{Negative}} \\
		\cmidrule(lr){2-3} \cmidrule(lr){4-5}
		& \textbf{Feature} & \textbf{$r$} & \textbf{Feature} & \textbf{$r$} \\
		\midrule
        \multirow[b]{5}{*}{Openness (U8)} 
		& resnet\_1573 & 0.30 & resnet\_334 & -0.25 \\
		& resnet\_1450 & 0.25 & resnet\_1322 & -0.22 \\
		& resnet\_1737 & 0.24 & resnet\_1881 & -0.21 \\
		& resnet\_947   & 0.24 & resnet\_1840 & -0.21 \\
		& resnet\_626   & 0.23 & resnet\_832  & -0.20 \\
		\cmidrule(lr){1-5}
		\multirow[b]{5}{*}{Conscientiousness (B24)} 
		& tfidf\_tb          & 0.29 & tfidf\_party   & -0.30 \\
		& tfidf\_celebrating & 0.26 & tfidf\_current & -0.24 \\
		& tfidf\_let s       & 0.25 & tfidf\_mo      & -0.21 \\
		& tfidf\_garden      & 0.24 & tfidf\_lost    & -0.21 \\
		& tfidf\_aking       & 0.23 & & \\
		\cmidrule(lr){1-5}
		\multirow[b]{5}{*}{Extraversion (B10)} 
		& w2v\_16  & 0.30 & w2v\_288 & -0.32 \\
		& w2v\_41  & 0.29 & w2v\_258 & -0.30 \\
		& w2v\_162 & 0.28 & w2v\_38  & -0.27 \\
		& w2v\_184 & 0.28 & w2v\_271 & -0.27 \\
		& w2v\_190 & 0.26 & w2v\_160 & -0.26 \\
		\cmidrule(lr){1-5}
		\multirow[b]{5}{*}{Agreeableness (M10)} 
		& w2v\_192            & 0.24 & w2v\_254 & -0.21 \\
		& w2v\_111            & 0.23 & w2v\_56  & -0.21 \\
		& w2v\_295            & 0.18 & w2v\_66  & -0.20 \\
		& w2v\_256            & 0.18 & w2v\_204 & -0.20 \\
		& w2v\_167            & 0.18 & w2v\_38  & -0.17 \\
		\cmidrule(lr){1-5}
		\multirow[b]{5}{*}{Neuroticism (TFIDF\_ALL)} 
		& tfidf\_siya      & 0.28 & tfidf\_credits   & -0.24 \\
		& tfidf\_ganito    & 0.28 & tfidf\_lights    & -0.24 \\
		& tfidf\_join      & 0.28 & tfidf\_heat      & -0.24 \\
		& tfidf\_anniversary & 0.28 & tfidf\_good food & -0.23 \\
		& tfidf\_facebook  & 0.25 & & \\
		\bottomrule
	\end{tabular}
\end{table}

% Table 2: Top 5 Features by Coefficient / Permutation Importance
\begin{table}[htbp]
	\centering
	\caption{Top 5 Features by Model Coefficient or Permutation Importance}
	\label{tab:importance}
	\begin{tabular}{@{} l l c @{}}
		\toprule
		\textbf{Trait (Model)} & \textbf{Feature} & \textbf{Value} \\
		\midrule
		\multirow{5}{3cm}{Openness (Logistic Regression --- Absolute Coef.)} 
		& resnet\_243  & 0.271 \\
		& resnet\_1751 & 0.269 \\
		& resnet\_1068 & 0.264 \\
		& resnet\_1545 & 0.261 \\
		& resnet\_547  & 0.254 \\
		\cmidrule(lr){1-3}
		\multirow{5}{3cm}{Conscientiousness (SVM Permutation --- Kappa Drop)} 
		& tfidf\_17       & 0.054 \\
		& tfidf\_amazing  & 0.054 \\
		& tfidf\_soon     & 0.039 \\
		& tfidf\_31       & 0.036 \\
		& tfidf\_love u   & 0.036 \\
		\cmidrule(lr){1-3}
		\multirow{5}{3cm}{Extraversion (Logistic Regression --- Absolute Coef.)} 
		& w2v\_w2v\_247 & 0.806 \\
		& w2v\_w2v\_210 & 0.763 \\
		& w2v\_w2v\_98  & 0.741 \\
		& w2v\_w2v\_242 & 0.728 \\
		& w2v\_w2v\_129 & 0.677 \\
		\cmidrule(lr){1-3}
		\multirow{5}{3cm}{Agreeableness (SVM Permutation --- Kappa Drop)} 
		& w2v\_w2v\_149 & 0.060 \\
		& w2v\_w2v\_168 & 0.048 \\
		& w2v\_w2v\_138 & 0.045 \\
		& w2v\_w2v\_15  & 0.045 \\
		& w2v\_w2v\_58  & 0.042 \\
		\cmidrule(lr){1-3}
		\multirow{5}{3cm}{Neuroticism (SVM Permutation --- Kappa Drop)} 
		& tfidf\_like s    & 0.033 \\
		& tfidf\_said      & 0.029 \\
		& tfidf\_mountains & 0.029 \\
		& tfidf\_house     & 0.026 \\
		& tfidf\_class     & 0.023 \\
		\bottomrule
	\end{tabular}
\end{table}


\subsubsection{Extraversion}

Extraversion emerged as the most predictable trait in this study ($\kappa = 0.3684$). The winning model (B10) utilized a combination of semantic text embeddings (Word2Vec) and aesthetic features (HSV), with Logistic Regression as the optimal algorithm.

\textbf{Key features:} Pearson correlation analysis revealed that high Extraversion is strongly linked to specific Word2Vec clusters (e.g., w2v\_16: $r = 0.30$, w2v\_41: $r = 0.29$) while showing negative correlations with others (e.g., w2v\_288: $r = -0.32$, w2v\_258: $r = -0.30$). Permutation importance further confirmed that Word2Vec dimensions dominate the model's decisions.

\begin{comment}
	conte\textbf{Interpretation:} These patterns suggest that the way social interaction and ``outgoing'' behavior are phrased in social media provides a more robust signal than visual data alone for this trait. The positive correlations likely capture language around social activities and engagement, while negative correlations may reflect introspective or solitary content.nt...
\end{comment}


\textbf{Connection to prior work:} This finding aligns with foundational work by Mairesse et al. (2007), who demonstrated that Extraversion is strongly correlated with specific linguistic categories—particularly words related to social processes and positive emotion. More recently, \citet{Ferwerda2018}) found that HSV features showed modest correlations with Extraversion. HSV alone achieved only $\kappa = 0.0115$, but when combined with Word2Vec, performance substantially improved. This supports Skowron et al.'s (2016) assertion that multimodal fusion—specifically combining linguistic and visual cues—yields optimal results for Extraversion, as the trait manifests through both what users say and the aesthetic quality of what they post.

\subsubsection{Openness}

Openness achieved its best result ($\kappa = 0.2762$) through the U8 combination, which is purely visual (ResNet50 + HSV) with Logistic Regression.

\textbf{Key features:} Model coefficients are dominated by deep visual features (e.g., resnet\_243: coefficient 0.271, resnet\_1751: 0.269, resnet\_1068: 0.264). Pearson correlations show positive associations with multiple ResNet dimensions (resnet\_1573: $r = 0.30$, resnet\_1450: $r = 0.25$) and negative correlations with others (resnet\_334: $r = -0.25$, resnet\_1322: $r = -0.22$).

\begin{comment}
	cont\textbf{Interpretation:} Higher Openness scores correlate with more complex visual information and specific color distributions, confirming that users high in Openness tend to post diverse and aesthetically varied content rather than conforming to a single ``visual brand.'' The aesthetic features, though lower-dimensional, provide interpretability: the pattern suggests openness manifests through visual exploration and novelty-seeking in image selection.ent...
\end{comment}


\textbf{Connection to prior work:} These results strongly align with Ferwerda et al.'s (2016) finding that Openness is reliably predicted from Instagram image features, particularly color diversity and compositional complexity.The dominance of visual over textual features for Openness confirms Azucar et al.'s (2018) meta-analytic finding that Openness is primarily expressed through ``demographic and aesthetic statistics'' rather than linguistic content, distinguishing it from traits like Conscientiousness that rely more heavily on text.

\subsubsection{Conscientiousness}

For Conscientiousness, the B24 combination ($\kappa = 0.2451$) demonstrated the importance of merging linguistic features with account behavioral metadata, with SVM as the optimal algorithm.

\textbf{Key features:} Permutation importance identified TF-IDF tokens as the most influential predictors, including \textit{tfidf\_17} (kappa drop: 0.054), \textit{amazing} (0.054), \textit{soon} (0.039), and \textit{love u} (0.036). Pearson correlation analysis reveals interpretable linguistic markers: positive correlations with tokens such as \textit{tb} ($r = 0.29$), \textit{celebrating} ($r = 0.26$), \textit{let s} ($r = 0.25$), and \textit{garden} ($r = 0.24$), while showing negative correlations with \textit{party} ($r = -0.30$), \textit{current} ($r = -0.24$), and \textit{mo} ($r = -0.21$).

\begin{comment}
	content\textbf{Interpretation:} The positive correlations suggest a pattern of posts centered on milestone reporting, gratitude, and organized activities—consistent with the trait's association with discipline and planfulness. The strong negative correlation with \textit{party} implies that users high in Conscientiousness may refrain from posting content associated with high-arousal, impulsive social activities, preferring more ``orderly'' or reflective content. The inclusion of metadata (post and following counts) further captures behavioral regularity—consistent posting patterns that content analysis alone cannot model....
\end{comment}


\textbf{Connection to prior work:} These findings resonate with Pennebaker and King's (1999) early work linking Conscientiousness to linguistic markers of achievement and organization. More specifically, our results align with Tighe and Cheng's (2018) pioneering Filipino APR study, which found that TF-IDF unigrams were the strongest predictors of Conscientiousness among Filipino Twitter users—a pattern our Instagram-based results replicate. The negative correlation with \textit{party} echoes Majumder et al.'s (2017) finding that Conscientious individuals use fewer words related to impulsive activities and leisure. The metadata contribution extends Azucar et al.'s (2018) observation that ``activity statistics'' (post frequency, following patterns) predict Conscientiousness, as organized behavior manifests not just in what users post, but in how regularly they engage with the platform.

\subsubsection{Agreeableness}

The best model for Agreeableness (M10, $\kappa = 0.1979$) relied on Word2Vec, HSV, and Metadata with SVM.

\textbf{Key features:} Permutation importance shows Word2Vec dimensions dominating (w2v\_149: kappa drop 0.060, w2v\_168: 0.048, w2v\_138: 0.045). Pearson correlations reveal positive associations with Word2Vec clusters (w2v\_192: $r = 0.24$, w2v\_111: $r = 0.23$) and negative correlations with others (w2v\_254: $r = -0.21$, w2v\_56: $r = -0.21$). An interpretable aesthetic finding is the positive correlation with \textbf{Aesthetic Value Mean} (brightness).

\begin{comment}
	conte\textbf{Interpretation:} The brightness correlation suggests that Agreeable users tend to post images that are brighter and more visually ``open''—aligning with psychological profiles of the trait involving optimism, warmth, and approachability. The Word2Vec patterns likely capture language around cooperation, positive social interactions, and empathetic expression.nt...
\end{comment}


\textbf{Connection to prior work:} Our Word2Vec results extend Mairesse et al.'s (2007) finding that Agreeableness is linguistically marked by words with "positive emotion" and "social processes". The metadata contribution mirrors Secuya's (2021) observation that account-level features (following counts, posting frequency) provide complementary signals for Agreeableness, as the trait manifests in social connectedness and engagement patterns. However, our modest performance ($\kappa = 0.1979$) suggests that Agreeableness remains challenging to predict from social media alone—a limitation noted across multiple Filipino APR studies (Tighe et al., 2022; Gomez et al., 2024).

\subsubsection{Neuroticism}

Neuroticism was best predicted using unimodal TF-IDF features ($\kappa = 0.1615$) with SVM—a significant finding, as it suggests that for this difficult-to-predict trait, specific keywords are more informative than visual or metadata patterns.

\textbf{Key features:} Permutation importance identifies tokens such as \textit{like s} (kappa drop: 0.033), \textit{said} (0.029), \textit{mountains} (0.029), \textit{house} (0.026), and \textit{class} (0.023) as most influential. Pearson correlations reveal Filipino-specific tokens as strong positive markers: \textit{siya} (third person pronoun, $r = 0.28$), \textit{ganito} (like this, $r = 0.28$), \textit{join} ($r = 0.28$), and \textit{anniversary} ($r = 0.28$). Negative correlations include \textit{credits} ($r = -0.24$), \textit{lights} ($r = -0.24$), \textit{heat} ($r = -0.24$), and \textit{good food} ($r = -0.23$).

\begin{comment}
	co\textbf{Interpretation:} The prominence of Filipino-specific pronouns (\textit{siya}) and demonstratives (\textit{ganito}) suggests that Neuroticism manifests in the structural and referential aspects of language rather than explicit emotional content. The positive correlation with \textit{anniversary} alongside negative correlation with \textit{good food} implies that high-Neuroticism users may post about personally significant events (potentially with anxious reflection) while avoiding content expressing simple pleasures or satisfaction—consistent with the trait's association with negative affect and rumination.
	ntent...
\end{comment}

\textbf{Connection to prior work:}  Our results align with Han et al. (2020), who demonstrated that interpretable word-level features often outperform dense embeddings for Neuroticism precisely because the trait manifests in specific lexical choices rather than broad semantic patterns. The presence of Filipino-specific markers echoes Tighe and Cheng's (2018) observation that code-switching and local language use carry personality-relevant signals that generic embeddings may miss. The overall low performance ($\kappa = 0.1615$) is consistent across Filipino APR literature—Acorda et al. (2019) achieved only $R^2 = 0.500$ (equivalent to modest classification performance), and Tighe et al. (2022) noted Neuroticism as the most challenging trait across both Twitter and Instagram subsets. This suggests that either Neuroticism is less visibly expressed on social media, or that the cues are more subtle and context-dependent than current feature sets capture.

\subsection{Summary of Observations}
The analysis confirms that textual modalities (TF-IDF and Word2Vec) are the primary drivers for Conscientiousness, Agreeableness, and Neuroticism, while visual modalities (ResNet and HSV) dominate Openness and Extraversion. Furthermore, the inclusion of Metadata acted as a crucial “refinement” feature for Agreeableness and Conscientiousness, providing a boost in Kappa by capturing behavioral regularity that content alone could not model.
\begin{comment}
\subsection{Stability of Multimodal Fusion}
In this analysis, stability is defined as the consistency of the model's predictive power across different trait-feature combinations, characterized by a narrower range of test kappa fluctuations and a higher frequency of positive results. 

The unimodal and bimodal visual sets (U7–U10) exhibited high volatility, with test kappa values swinging from 0.2762 to -0.1982. In contrast, Multimodal combinations (M1–M5) demonstrated greater stability. While they did not always reach the absolute highest peaks found in specialized visual sets, they maintained a more consistent positive range—predominantly between 0.05 and 0.24 for Openness, Conscientiousness, and Extraversion—and significantly reduced the frequency of extreme negative outliers compared to unimodal runs.


	content...

\subsection{Algorithm Comparison}
The three models were evaluated across 15 different feature combinations. Their performance is summarized below:

\begin{itemize}
	\item \textbf{XGBoost:} Top performer in 6 out of 15 combinations, primarily dominating in Extraversion and Openness. It proved most effective at handling non-linear relationships within bimodal visual data. However, it was also the most prone to overfitting on high-dimensional multimodal sets, where validation kappa reached 0.8372 while the test kappa remained negative.
	
	\item \textbf{Logistic Regression:} Most effective for Conscientiousness and Agreeableness, leading in 5 out of 15 combinations. It performed best when utilizing Metadata and HSV features, suggesting these features share a more linear relationship with "orderly" personality traits.
	
	\item \textbf{SVM:} Best in 4 out of 15 combinations, specifically those involving Multimodal Fusion (M1, M2, M4, and M5). It achieved the highest overall score for Extraversion ($\kappa = 0.2456$). SVM demonstrated a superior ability to find optimal hyperplanes in the high-dimensional space created when combining linguistic features with deep visual data.
\end{itemize}
\end{comment}
