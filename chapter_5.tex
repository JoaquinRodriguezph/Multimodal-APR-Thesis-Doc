\chapter{Results and Discussion}
\label{sec:results}

\section{Results}
\begin{table}[H]
    \centering
    \caption{Best performing models per personality trait with its feature combination}
    \label{tab:best_models}
    \begin{tabularx}{\textwidth}{l|X|X|l}
        \hline
        \textbf{Trait} & \textbf{Feature Type} & \textbf{Model} & \textbf{Test Kappa} \\ \hline
        Openness & Resnet50 + HSV & Logistic Regression (GPU) & 0.2762 \\ \hline
        Conscientiousness & TFIDF + Word2Vec, Metadata & SVM (GPU) & 0.2451 \\ \hline
        Extraversion & Word2Vec, HSV & Logistic Regression (GPU) & 0.3684 \\ \hline
        Agreeableness & Word2Vec, Metadata & SVM (GPU) & 0.2129 \\ \hline
        Neuroticism & Word2Vec, HSV & XGBoost & 0.0749 \\ \hline
    \end{tabularx}
\end{table}

In the Table \ref{tab:best_models}, among the personality traits, Extraversion has Logistic Regression, run in GPU, as the best performing model with the highest kappa score, boasting 0.3684, with the feature combination Word2Vec and HSV. On the other hand, Neuroticism has XGBoost as the worst performing model compared to the models of each trait, with the kappa score of 0.0749, with the feature combination Word2Vec and HSV.

Openness, Extraversion and Neuroticism are consistent with their feature type combination, being Word2Vec and HSV. However, Conscientiousness only has TF-IDF, Word2Vec, and Metadata as the feature type combination used by the best model, which is SVM, under Conscientiousnes with the kappa score of 0.2451. Agreeableness has Word2Vec and Metadata combination, which are the text feature and a metadata feature.

\begin{table}[h]
	\centering
	\caption{Unimodal Text Results (Cohen's $\kappa$)}
	\label{tab:unimodal_text_performance}
	\begin{tabular}{llccccc}
		\hline
		\textbf{Feature} & \textbf{Variant} & \textbf{O} & \textbf{C} & \textbf{E} & \textbf{A} & \textbf{N} \\
		\hline
		\multirow{2}{*}{TF-IDF} % Added {2} and {*}
		& 1.5k & 0.1005 & 0.0969 & 0.1754 & 0.1217 & -0.0242 \\
		& All  & 0.1531 & 0.2271 & 0.2456 & 0.0200 & 0.1615 \\
		\hline
		\multirow{2}{*}{Word2Vec} % Added {2} and {*}
		& WWE & 0.1381 & 0.0297 & 0.1754 & 0.1174 & 0.0000 \\
		& PWE & 0.2267 & 0.1102 & 0.3684 & 0.1620 & -0.0447 \\
		\hline
		Combined & -- & 0.1706 & 0.1960 & 0.1404 & 0.1206 & 0.0188 \\
		\hline
	\end{tabular}
\end{table}

In Table ~\ref{tab:unimodal_text_performance}, in the case of TF-IDF, using all tags rather than only the ones that fall in the 1,500 most common generally performed better, with the exception of Agreeability. This shows that using all available information and taking less common nuances into account likely improves personality prediction. Between Word2Vec’s two averaging strategies, the one that prioritized post composition (“Posts Weighed Equally”) performed better than the alternative, except for Neuroticism; however, the difference for that trait is minor. This could mean that taking overall post composition and sentiment into account, rather than just individual word usage, may reveal aspects of personality that are useful in its prediction. 

\begin{table}[H]
    \centering
    \caption{Best performing model after learning unimodal features in terms of Cohen's Kappa running on a testing data, set in a specific combination of features per personality trait}
    \label{tab:uni_kappa}
    \begin{tabularx}{\textwidth}{l|X|X|l}
    \hline
        \textbf{Trait} & \textbf{Combination} & \textbf{Model} & \textbf{Test Kappa} \\ \hline
        Openness & Resnet50 + HSV & Logistic Regression (GPU) & 0.2762 \\ \hline
        Conscientiousness & Resnet50 + Imagga & XGBoost & 0.1174 \\ \hline
        Extraversion & Resnet50 + HSV & XGBoost & 0.2632 \\ \hline
        Agreeableness & Resnet50 + HSV & SVM (GPU) & 0.0444 \\ \hline
        Neuroticism & Resnet50 + HSV & SVM (GPU) & 0.0037 \\ \hline
    \end{tabularx}
    \label{bestcohenmodel}
\end{table}

In Table ~\ref{tab:uni_kappa}, Openness has Logistic Regression as the best model that used ResNet50 and HSV with the test kappa of 0.2762. Then, Neuroticism has very near zero test kappa of 0.0037 for SVM that uses Resnet50 and HSV.

All of the traits in Table \ref{tab:uni_kappa} except Consciousness have Resnet50 and HSV as the best feature combination, while Resnet50 and Imagga performed the best for Consciousness. Openness, Conscientiousness, and Extraversion have best performing models with test kappa scores that exceed 0.1 threshold, while Agreeableness and Neuroticism have their best models perform weak with their test kappa less than 0.1. Neuroticism has the best model with the least scores among the other personality traits.

Aside from Logistic Regression, two models occur twice among the personality traits. Conscientiousness and Extraversion have XGBoost as the best performing model, while Agreeable and Neuroticism have SVM.

\begin{table}[H]
    \centering
    \caption{Best performing model after learning bimodal features in terms of Cohen's Kappa run on a testing data set in a specific combination of features per personality trai}
    \label{tab:bi_kappa}
    \begin{tabularx}{\textwidth}{l|X|X|l}
    \hline
        \textbf{Trait} & \textbf{Combination} & \textbf{Model} & \textbf{Test Kappa} \\ \hline
        Openness & Resnet50 + HSV, Metadata & Logistic Regression (GPU) & 0.2602 \\ \hline
        Conscientiousness & TFIDF + Word2Vec, Metadata & SVM (GPU) & 0.2451 \\ \hline
        Extraversion & Word2Vec, HSV & Logistic Regression (GPU) & 0.3684 \\ \hline
        Agreeableness & Word2Vec, Metadata & SVM (GPU) & 0.2129 \\ \hline
        Neuroticism & Word2Vec, HSV & XGBoost & 0.0749 \\ \hline
    \end{tabularx}
    \label{bestcohenmodel}
\end{table}

In Table \ref{tab:bi_kappa}, Extraversion has Logistic Regression as the best performing model with the test kappa of 0.3684 among the traits, which uses Word2Vec and HSV. Neuroticism has XGBoost as the worst performing model with the test kappa of 0.0749 among the traits, which uses Word2Vec and HSV. Openness, Conscientiousness and Agreeable have test kappas close to each other with 0.2602, 0.2451 and 0.2129, respectively. Openness was Logistic  Regression as the best performing model.

For feature type combination, Extraversion and Neuroticism have the same combination with Word2Vec and HSV in Table \ref{tab:bi_kappa}, except Extraversion has Logistic Regression and Neuroticism has XGBoost. The rest of the traits have unique feature combination: Openness has Resnet50, HSV and metadata; Conscientiousness has TF-IDF, Word2Vec, and metadata; and Agreeable has Word2Vec and Metadata. 

The models except XGBoost appear at least two times in Table \ref{tab:bi_kappa}, which are Logistic Regression and SVM. Openness and Extraversion have Logistic Regression as the best model, while Conscientiousness and Agreeable have SVM as the best model.

\begin{table}[H]
    \centering
    \caption{Best performing model after learning multimodal features in terms of Cohen's Kappa run on a testing data set in a specific combination of features per personality trait for each modality}
    \label{tab:mm_kappa}
    \begin{tabularx}{\textwidth}{l|X|l|l}
    \hline
        \textbf{Trait} & \textbf{Combination} & \textbf{Model} & \textbf{Test Kappa} \\ \hline
        Openness & Word2Vec, Resnet50 + Imagga + HSV, Metadata & XGBoost & 0.2267 \\ \hline
        Conscientiousness & TFIDF + Word2Vec, HSV, Metadata & SVM (GPU) & 0.2442 \\ \hline
        Extraversion & TFIDF + Word2Vec, Resnet50, Metadata & XGBoost & 0.2807 \\ \hline
        Agreeableness & Word2Vec, HSV, Metadata & SVM (GPU) & 0.1979 \\ \hline
        Neuroticism & Word2Vec, HSV, Metadata & XGBoost & 0.065 \\ \hline
    \end{tabularx}
    \label{bestcohenmodel}
\end{table}

In Table \ref{tab:mm_kappa}, Extraversion has SVM as the best performing model that used TF-IDF, Word2Vec, Resnet50 and metadata with the kappa score of 0.2807, while Neuroticism has XGBoost as the worst performing model that used Word2Vec, HSV and metadata with the kappa score of 0.065.

Across all personality traits, Word2Vec and metadata appear all times in Table \ref{tab:mm_kappa}. HSV is the next most occurring, where only Extraversion does not use HSV. Then TF-IDF comes in to appear in Conscientiousnes and Extraversion. Then ResNet50 is used in Openness and Extraversion. Lastly, Imagga appears once in Openness.

XGBoost is used three times in Table \ref{tab:mm_kappa}, whereas SVM appears twice in the same table. Openness, Extraversion and Neuroticism use XGBoost as the best performing model, while Conscientiousness and Agreeableness use SVM also as the best performing model.

\section{Discussion}

\subsection{Multimodality}
When combining modalities, the performance gains are generally limited and inconsistent relative to unimodal performance; however, certain multimodal configurations do demonstrate meaningful improvements over their individual components. Several bimodal models outperform their respective single-modality counterparts. For instance, the Word2Vec + HSV combination for Conscientiousness achieves a kappa of 0.2233, exceeding the performance of Word2Vec alone ($\kappa$ = 0.1102) and HSV alone ($\kappa$ = 0.0115). This is one of several cases in which bimodal feature fusion leads to improved performance, suggesting that complementary personality signals emerge when modalities are combined. This effect is most clearly illustrated by two models: (1) the TF-IDF + Word2Vec + metadata combination and (2) the Word2Vec + metadata combination. These models achieve the best performance for Conscientiousness ($\kappa$ = 0.2451) and Agreeableness ($\kappa$ = 0.2129), respectively.

Trimodal models exhibited patterns similar to bimodal configurations: performance improvements were observed with the addition of modalities, but the gains were generally modest. One notable example is the Word2Vec + ResNet + Imagga combination, whose kappa increased from 0.1053 to 0.2105 when metadata was incorporated, representing a 0.1052 improvement. However, unlike the bimodal results, none of the trimodal models achieved the best performance for any personality trait in this study. This outcome may reflect limitations in the specific feature combinations explored rather than the ineffectiveness of higher modality fusion, as increasing modality count has already demonstrated potential for performance gains.


\subsection{Trait and feature performance}
An examination of the top models for each trait reveals distinct modality patterns across personality dimensions. For Openness, ResNet-based features clearly dominate, as the highest-performing models consist either of ResNet alone or ResNet combined with metadata and HSV features. Given that ResNet contributes 2,048 features, compared to only 21 from HSV and 2 from metadata, these multimodal configurations are likely driven primarily by the high-dimensional visual representations. This interpretation is further supported by the fact that unimodal ResNet achieves the strongest performance for this trait. The best-performing non-ResNet model is unimodal Word2Vec ($\kappa$ = 0.2267), which also appears frequently among the top Openness models below the ResNet-dominated group. This pattern suggests that while visual representations are the primary signal for Openness in this dataset, semantic textual features provide a secondary yet meaningful contribution.

For Conscientiousness, the highest score comes from the bimodal TF-IDF + Word2Vec + metadata model ($\kappa$ = 0.2451). Most top configurations involve combinations of TF-IDF, Word2Vec, metadata, and HSV, with at least one textual representation always present.  Although metadata and HSV have relatively low dimensionality, their frequent appearance among high-performing configurations instead of the counterpart configurations that lack them suggests that they contribute useful complementary information rather than merely adding noise. Additionally, given the substantially higher dimensionality of textual features, it can be concluded that Conscientiousness is primarily text-driven, with metadata and HSV acting as supporting signals rather than dominant ones. Compared to Openness, image features are largely absent from the top models.

Extraversion is the overall best-performing trait, with unimodal Word2Vec (“Posts Weighted Equally”) achieving the highest kappa score of 0.3684. Similar to the Openness–ResNet pattern, Word2Vec performs best independently and is followed by combinations with HSV and metadata. ResNet also appears in several high-ranking combinations after the strongest Word2Vec-based models, indicating that visual features can provide additional gains. TF-IDF appears occasionally among top models but generally trails behind Word2Vec. 

For Agreeableness, the top-performing model is Word2Vec + metadata ($\kappa$ = 0.2129). Word2Vec appears in nearly all top configurations, typically combined with metadata, TF-IDF, or HSV. This pattern closely resembles Conscientiousness, where text forms the backbone of performance, and lower-dimensional features such as metadata and HSV frequently accompany it. As with Conscientiousness, these smaller feature sets appear to provide complementary value despite their limited dimensionality.

In contrast, Neuroticism shows a markedly different pattern. The best-performing model is unimodal TF-IDF ($\kappa$ = 0.1615), substantially outperforming the next highest model, Word2Vec + HSV ($\kappa$ = 0.0749). Beyond this leading result, most models cluster near zero. This trait has proven difficult to predict, no matter the combination of features or modalities.

From these rankings, some general observations can be made. Metadata and HSV frequently appear in high-performing combinations despite their low dimensionality. This raises two possibilities: either they add minimal noise and simply “ride along” with stronger high-dimensional features, or they provide compact but meaningful complementary signals. The substantial $\kappa$ increase observed in earlier experiments when adding only the two metadata features (+0.1052) supports the latter interpretation in at least some cases. Conversely, Imagga—despite having the largest dimensionality (36,398 features)—rarely appears among top-performing models. This suggests that high dimensionality alone does not guarantee predictive value and may instead introduce noise that limits its contribution. It may also suggest that Imagga does not carry a valuable amount of personality data.


\section{Model Performance Analysis}
The performance of the classification models—Logistic Regression (LR) and Support Vector Machine (SVM)—was evaluated to determine the best configurations for each of the Big Five personality traits. The primary metric for evaluation was Cohen’s Kappa ($\kappa$), which measures the agreement between predicted and actual labels while accounting for chance.

\subsection{Overall Best Performing Models}
Table~\ref{tab:best_overall} summarizes the winning configurations for each trait. Unlike earlier broad sweeps, these models represent the optimal balance of modalities and algorithm selection.

\begin{table}[H]
	\centering
	\caption{Best Performing Models and Hyperparameters per Personality Trait}
	\label{tab:best_overall}
	\begin{tabularx}{\textwidth}{@{} l X X c @{}}  % X column allows wrapping in Model column
		\toprule
		\textbf{Trait} & \textbf{Combination} & \textbf{Model (Params)} & \textbf{Kappa} \\
		\midrule
		O          & ResNet50 + HSV                         & LR ($C=0.1$)                                  & 0.2762 \\
		C & TF-IDF + W2V + Metadata                 & SVM ($C=1.0$, $\gamma=0.0001$, RBF)           & 0.2451 \\
		E      & Word2Vec + HSV                           & LR ($C=0.3594$)                               & 0.3684 \\
		A      & Word2Vec + HSV + Metadata                & SVM ($C=10.0$, $\gamma=0.0001$, RBF)          & 0.1979 \\
		N       & TF-IDF                      & SVM ($C=1.0$, $\gamma=0.0001$, RBF)           & 0.1615 \\
		\bottomrule
	\end{tabularx}
\end{table}

\subsection{Trait-Specific Analysis and Interpretable Features}
% Table 1: Top Pearson Correlations by Trait
\begin{table}[htbp]
	\centering
	\caption{Top Pearson Correlations by Trait (positive and negative)}
	\label{tab:pearson}
	\begin{tabular}{@{} l l c l c @{}}
		\toprule
		\textbf{Trait} & \multicolumn{2}{c}{\textbf{Positive}} & \multicolumn{2}{c}{\textbf{Negative}} \\
		\cmidrule(lr){2-3} \cmidrule(lr){4-5}
		& \textbf{Feature} & \textbf{$r$} & \textbf{Feature} & \textbf{$r$} \\
		\midrule
        \multirow[b]{5}{*}{Openness (U8)} 
		& resnet\_1573 & 0.30 & resnet\_334 & -0.25 \\
		& resnet\_1450 & 0.25 & resnet\_1322 & -0.22 \\
		& resnet\_1737 & 0.24 & resnet\_1881 & -0.21 \\
		& resnet\_947   & 0.24 & resnet\_1840 & -0.21 \\
		& resnet\_626   & 0.23 & resnet\_832  & -0.20 \\
		\cmidrule(lr){1-5}
		\multirow[b]{5}{*}{Conscientiousness (B24)} 
		& tfidf\_tb          & 0.29 & tfidf\_party   & -0.30 \\
		& tfidf\_celebrating & 0.26 & tfidf\_current & -0.24 \\
		& tfidf\_let s       & 0.25 & tfidf\_mo      & -0.21 \\
		& tfidf\_garden      & 0.24 & tfidf\_lost    & -0.21 \\
		& tfidf\_aking       & 0.23 & & \\
		\cmidrule(lr){1-5}
		\multirow[b]{5}{*}{Extraversion (B10)} 
		& w2v\_16  & 0.30 & w2v\_288 & -0.32 \\
		& w2v\_41  & 0.29 & w2v\_258 & -0.30 \\
		& w2v\_162 & 0.28 & w2v\_38  & -0.27 \\
		& w2v\_184 & 0.28 & w2v\_271 & -0.27 \\
		& w2v\_190 & 0.26 & w2v\_160 & -0.26 \\
		\cmidrule(lr){1-5}
		\multirow[b]{5}{*}{Agreeableness (M10)} 
		& w2v\_192            & 0.24 & w2v\_254 & -0.21 \\
		& w2v\_111            & 0.23 & w2v\_56  & -0.21 \\
		& w2v\_295            & 0.18 & w2v\_66  & -0.20 \\
		& w2v\_256            & 0.18 & w2v\_204 & -0.20 \\
		& w2v\_167            & 0.18 & w2v\_38  & -0.17 \\
		\cmidrule(lr){1-5}
		\multirow[b]{5}{*}{Neuroticism (TFIDF\_ALL)} 
		& tfidf\_siya      & 0.28 & tfidf\_credits   & -0.24 \\
		& tfidf\_ganito    & 0.28 & tfidf\_lights    & -0.24 \\
		& tfidf\_join      & 0.28 & tfidf\_heat      & -0.24 \\
		& tfidf\_anniversary & 0.28 & tfidf\_good food & -0.23 \\
		& tfidf\_facebook  & 0.25 & & \\
		\bottomrule
	\end{tabular}
\end{table}

% Table 2: Top 5 Features by Coefficient / Permutation Importance
\begin{table}[htbp]
	\centering
	\caption{Top 5 Features by Model Coefficient or Permutation Importance}
	\label{tab:importance}
	\begin{tabular}{@{} l l c @{}}
		\toprule
		\textbf{Trait (Model)} & \textbf{Feature} & \textbf{Value} \\
		\midrule
		\multirow{5}{3cm}{Openness (Logistic Regression --- Absolute Coef.)} 
		& resnet\_243  & 0.271 \\
		& resnet\_1751 & 0.269 \\
		& resnet\_1068 & 0.264 \\
		& resnet\_1545 & 0.261 \\
		& resnet\_547  & 0.254 \\
		\cmidrule(lr){1-3}
		\multirow{5}{3cm}{Conscientiousness (SVM Permutation --- Kappa Drop)} 
		& tfidf\_17       & 0.054 \\
		& tfidf\_amazing  & 0.054 \\
		& tfidf\_soon     & 0.039 \\
		& tfidf\_31       & 0.036 \\
		& tfidf\_love u   & 0.036 \\
		\cmidrule(lr){1-3}
		\multirow{5}{3cm}{Extraversion (Logistic Regression --- Absolute Coef.)} 
		& w2v\_w2v\_247 & 0.806 \\
		& w2v\_w2v\_210 & 0.763 \\
		& w2v\_w2v\_98  & 0.741 \\
		& w2v\_w2v\_242 & 0.728 \\
		& w2v\_w2v\_129 & 0.677 \\
		\cmidrule(lr){1-3}
		\multirow{5}{3cm}{Agreeableness (SVM Permutation --- Kappa Drop)} 
		& w2v\_w2v\_149 & 0.060 \\
		& w2v\_w2v\_168 & 0.048 \\
		& w2v\_w2v\_138 & 0.045 \\
		& w2v\_w2v\_15  & 0.045 \\
		& w2v\_w2v\_58  & 0.042 \\
		\cmidrule(lr){1-3}
		\multirow{5}{3cm}{Neuroticism (SVM Permutation --- Kappa Drop)} 
		& tfidf\_like s    & 0.033 \\
		& tfidf\_said      & 0.029 \\
		& tfidf\_mountains & 0.029 \\
		& tfidf\_house     & 0.026 \\
		& tfidf\_class     & 0.023 \\
		\bottomrule
	\end{tabular}
\end{table}


\subsubsection{Extraversion}
Extraversion emerged as the most predictable trait in this study ($\kappa = 0.3684$). The winning model (B10) utilized a combination of semantic text embeddings (Word2Vec) and aesthetic features (HSV).

Analysis of Pearson correlations indicates that high Extraversion is strongly linked to specific semantic clusters in post captions. This suggests that the way social interaction and “outgoing” behavior are phrased in social media provides a more robust signal than visual data alone for this trait.

\subsubsection{Openness}
Openness achieved its best result ($\kappa = 0.2762$) through the U8 combination, which is purely visual (ResNet50 + HSV). While the model coefficients are dominated by deep visual features (ResNet), the aesthetic features provide a layer of interpretability. Higher Openness scores correlate with more complex visual information and specific color distributions, confirming that users high in Openness tend to post diverse and aesthetically varied content rather than conforming to a single “visual brand.”

\subsubsection{Conscientiousness}
For Conscientiousness, the B24 combination ($\kappa = 0.2451$) demonstrated the importance of merging linguistic features with account behavioral metadata.

The Pearson correlation analysis for B24 reveals highly interpretable linguistic markers:
\begin{itemize}
	\item \textbf{Positive Correlations:} Tokens such as \textit{grateful}, \textit{amazing}, \textit{experience}, and \textit{celebrating} show strong positive relationships with high Conscientiousness. This suggests a pattern of posts centered on milestone reporting and expressions of gratitude.
	\item \textbf{Negative Correlations:} The token \textit{party} shows a strong negative correlation. This implies that users high in Conscientiousness may refrain from posting content associated with high-arousal, impulsive social activities, preferring more “orderly” or reflective content.
\end{itemize}

\subsubsection{Agreeableness}
The best model for Agreeableness (M10, $\kappa = 0.1979$) relied on Word2Vec, HSV, and Metadata. An interpretable aesthetic finding for this trait is the positive correlation with \textbf{Aesthetic Value Mean} (brightness). This suggests that Agreeable users tend to post images that are brighter and more visually “open,” which aligns with psychological profiles of the trait involving optimism and approachability.

\subsubsection{Neuroticism}
Neuroticism was best predicted using unimodal TF-IDF features ($\kappa = 0.1615$). This is a significant finding, as it suggests that for this difficult-to-predict trait, specific keywords are more informative than visual or metadata patterns.

The top correlated features include Filipino-specific tokens such as \textit{siya} (third person pronoun) and \textit{ganito} (like this). Interestingly, high Neuroticism was positively correlated with tokens like \textit{anniversary}, but negatively correlated with \textit{good food}, suggesting that the trait manifests in specific contextual reflections rather than generic lifestyle content.

\subsection{Summary of Observations}
The analysis confirms that textual modalities (TF-IDF and Word2Vec) are the primary drivers for Conscientiousness, Agreeableness, and Neuroticism, while visual modalities (ResNet and HSV) dominate Openness and Extraversion. Furthermore, the inclusion of Metadata acted as a crucial “refinement” feature for Agreeableness and Conscientiousness, providing a boost in Kappa by capturing behavioral regularity that content alone could not model.
\begin{comment}
\subsection{Stability of Multimodal Fusion}
In this analysis, stability is defined as the consistency of the model's predictive power across different trait-feature combinations, characterized by a narrower range of test kappa fluctuations and a higher frequency of positive results. 

The unimodal and bimodal visual sets (U7–U10) exhibited high volatility, with test kappa values swinging from 0.2762 to -0.1982. In contrast, Multimodal combinations (M1–M5) demonstrated greater stability. While they did not always reach the absolute highest peaks found in specialized visual sets, they maintained a more consistent positive range—predominantly between 0.05 and 0.24 for Openness, Conscientiousness, and Extraversion—and significantly reduced the frequency of extreme negative outliers compared to unimodal runs.


	content...

\subsection{Algorithm Comparison}
The three models were evaluated across 15 different feature combinations. Their performance is summarized below:

\begin{itemize}
	\item \textbf{XGBoost:} Top performer in 6 out of 15 combinations, primarily dominating in Extraversion and Openness. It proved most effective at handling non-linear relationships within bimodal visual data. However, it was also the most prone to overfitting on high-dimensional multimodal sets, where validation kappa reached 0.8372 while the test kappa remained negative.
	
	\item \textbf{Logistic Regression:} Most effective for Conscientiousness and Agreeableness, leading in 5 out of 15 combinations. It performed best when utilizing Metadata and HSV features, suggesting these features share a more linear relationship with "orderly" personality traits.
	
	\item \textbf{SVM:} Best in 4 out of 15 combinations, specifically those involving Multimodal Fusion (M1, M2, M4, and M5). It achieved the highest overall score for Extraversion ($\kappa = 0.2456$). SVM demonstrated a superior ability to find optimal hyperplanes in the high-dimensional space created when combining linguistic features with deep visual data.
\end{itemize}
\end{comment}
