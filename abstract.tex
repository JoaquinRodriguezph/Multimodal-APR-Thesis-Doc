%%%%%%%%%%%%%%%%%%%%%%%%%%%%%%%%%%%%%%%%%%%%%%%%%%%%%%%%%%%%%%%%%%%%%%%%%%%%%%%%%%%%%%%%%%%%%%%%%%%%%%
%
%   Filename    : abstract.tex 
%
%   Description : This file will contain your abstract.
%                 
%%%%%%%%%%%%%%%%%%%%%%%%%%%%%%%%%%%%%%%%%%%%%%%%%%%%%%%%%%%%%%%%%%%%%%%%%%%%%%%%%%%%%%%%%%%%%%%%%%%%%%

\begin{abstract}
%% \begin{mdframed} [style=highlight]
Filipino Automatic Personality Recognition (APR) is a growing field focused on recognizing personality traits within the Filipino population. Current studies have primarily explored the textual modality, particularly content from X (formerly Twitter). However, prior research has shown that using both visual and linguistic cues jointly enhances trait prediction accuracy, and Instagram was identified as a viable source of multimodal data. This presents an opportunity to explore a multimodal approach in the context of Filipino APR using Instagram data in order to improve upon existing models. Given this, we propose an Instagram data-based multimodal APR framework for Filipino users, combining image features with bilingual (English/Tagalog) text analysis to advance beyond unimodal approaches. Our methodology employs classification models with intermediate attention-based fusion of image-text-metadata features, systematically comparing unimodal and multimodal performance across Big Five traits while adhering to ethical guidelines established in the PagkataoKo Dataset. This approach may identify an optimal data source for Filipino APR for consideration in future research.
%% \end{mdframed}




%
%  Do not put citations or quotes in the abract.
%
\end{abstract}
